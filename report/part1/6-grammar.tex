\subsection{Application: Grammars}\label{subsec:application}

The last section shows a concrete application of structured graphs:
a grammar analysis and transformations. They discuss 3 different operations
on grammars: \emph{nullability}, \emph{first set} and \emph{normalization}.

The grammar is defined as a mutually recursive collection of patterns,
where patterns can also refer to themselves or other patterns.

\vspace{1mm}
\begin{minted}[escapeinside=<>,mathescape=true, xleftmargin=10pt,fontsize=\small, style=emacs,samepage=true]{haskell}
data PatternF a
  = Term String -- Terminal
  | Epsilon -- Empty string
  | Seq a a -- Intersection
  | Alt a a -- Union
  deriving (Functor, Foldable, Traversable)
\end{minted}
\vspace{1mm}

\textbf{Nullability}\quad Nullability determines whether a given nonterminal can produce the empty string. The analysis is defined as follows:

\vspace{1mm}
\begin{minted}[escapeinside=<>,mathescape=true, xleftmargin=10pt,fontsize=\small, style=emacs,samepage=true]{haskell}
nullF :: PatternF Bool -> Bool
nullF (Term s)     = False
nullF Epsilon      = True
nullF (Seq g1 g2)  = g1 && g2
nullF (Alt g1 g2)  = g1 || g2

nullable :: Graph PatternF -> Bool
nullable = sfold nullF False
\end{minted}
\vspace{1mm}
