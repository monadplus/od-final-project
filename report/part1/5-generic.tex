\subsection{Generic Structured Graphs}\label{subsec:generic}

This section shows that by using some lightweight \emph{datatype-generic programming} techniques it
it possible to define highly reusable combinators for manipulating structured graphs of different types.

A generic datatype for structured graphs can be defined as follows:

\vspace{1mm}
\begin{minted}[escapeinside=<>,mathescape=true, xleftmargin=10pt,fontsize=\small, style=emacs,samepage=true]{haskell}
data Rec f a
  = Var a
  | Mu ([a] -> [f (Rec f a)])
  | In (f (Rec f a))

newtype Graph f = Hide {reveal :: forall a . Rec f a}
\end{minted}
\vspace{1mm}

\textbf{Streams Revisited}\quad To recover streams, the type-constructor $f$ is instantiated as follows:

\vspace{1mm}
\begin{minted}[escapeinside=<>,mathescape=true, xleftmargin=10pt,fontsize=\small, style=emacs,samepage=true]{haskell}
data StreamF a r = Cons a r
  deriving (Functor, Foldable, Traversable)

type Stream a = Graph (StreamF a)
\end{minted}
\vspace{1mm}

\textbf{Trees Revisited}\quad To recover cyclic trees the functor $f$ is instantated as follows:

\vspace{1mm}
\begin{minted}[escapeinside=<>,mathescape=true, xleftmargin=10pt,fontsize=\small, style=emacs,samepage=true]{haskell}
data TreeF a r = Empty | Fork a r r
  deriving (Functor, Foldable, Traversable)

type Tree a = Graph (TreeF a)
\end{minted}
\vspace{1mm}
