\subsection{Structured Graphs}\label{subsec:graphs}

Structured graphs use the recursive binders introduced in Section ref{sec:recursive} to describe cyclic structures. We consider two types of cyclic structures: \emph{cyclic streams} and \emph{cyclic binary trees}.

\subsubsection{Cyclic Streams and Back Edges}
\label{subsubsec:streams}

A datatype for cyclic streams can be defined as follows:

\vspace{1mm}
\begin{minted}[escapeinside=<>,mathescape=true, xleftmargin=10pt,fontsize=\small, style=emacs]{haskell}
data PStream a v
  = Var v
  | Mu (v -> PStream a v)
  | Cons a (PStream a v)
newtype Stream a = HideStream {revealStream :: <$\forall v.$> PStream a v}
\end{minted}
\vspace{1mm}

\subsubsection{Cyclic Binary Trees and Cross Edges}
\label{subsubsec:trees}

The datatype for cyclic binary trees is as follows:

\vspace{1mm}
\begin{minted}[escapeinside=<>,mathescape=true, xleftmargin=10pt,fontsize=\small, style=emacs,samepage=true]{haskell}
data PTree a v
  = Var v
  | Mu ([v] -> [PTree a v])
  | Empty
  | Fork a (PTree a v) (PTree a v)
newtype Tree a = HideTree {revealTree :: forall v . PTree a v}
\end{minted}
\vspace{1mm}

The main difference to the datatype of streams (besides the tree-specific constructors \code{Empty} and \code{Fork}) is the need for the recursive multi-binder to model cross edges.
