\section{Parametric HOAS}\label{sec:phoas}

Parametric Higher-Order Abstract Syntax (PHOAS) has a unique combination of advantages:

\begin{itemize}
  \item Guaranteed \emph{well-scopedness}.
  \item No explicit manipulation of environments.
  \item Easy to define operations.
\end{itemize}

To illustrate the advantages of PHOAS in more detail, they used the lambda calculus presented in Listing \ref{lst:calculus}.

\textbf{Well-scopedness}\quad The type argument $\alpha$ in $PLambda \ \alpha$ is supposed to be abstract. To enforce this, a universal quantifier is used in the definition. Using $\alpha$ as an abstract type ensures that only variables \emph{bound} by a constructor $Lam$ can be used in the constructor $Var$. For example, the \emph{identity function} can be defined as \mintinline[style=emacs,fontsize=\footnotesize]{haskell}{idLambda = PLam (\x -> PVar x)}.

\textbf{No explicit manipulation of environments}\quad Functions defined over PHOAS-based representations avoid the need for explicit manipulation of environments carrying the binding information. Instead, environments are implicitly handled by the meta-language. % The evaluator for our lambda calculus presented in Figure \ref{lst:evaluator} illustrates this.

\begin{listing}[H]
\begin{minted}[escapeinside=<>,mathescape=true, xleftmargin=10pt,fontsize=\small, style=emacs]{haskell}
data PLambda a
= PVar a
| PLit Int
| PBool Bool
| PIf (PLambda a) (PLambda a) (PLambda a)
| PAdd (PLambda a) (PLambda a)
| PMult (PLambda a) (PLambda a)
| PEq (PLambda a) (PLambda a)
| PLam (a -> PLambda a)
| PApp (PLambda a) (PLambda a)

newtype Lambda = forall a. PLambda a
\end{minted}
\caption{PHOAS-encoded lambda calculus}
\label{lst:calculus}
\end{listing}

% \begin{listing}[H]
% \begin{minted}[escapeinside=<>,mathescape=true, xleftmargin=10pt,fontsize=\small, style=emacs]{haskell}
% data Value
%   = VLit Int
%   | VBool Bool
%   | Val (Value -> Value)

% eval :: Lambda -> Value
% eval e = evalPLambda (revealLambda e)
%   where
%     evalPLambda :: PLambda Value -> Value
%     evalPLambda (PVar v) = v
%     evalPLambda (PLit n) = VLit n
%     evalPLambda (PBool b) = VBool b
%     evalPLambda (PIf e1 e2 e3) = case evalPLambda e1 of
%       VBool b -> if b then evalPLambda e2 else evalPLambda e3
%     evalPLambda (PAdd e1 e2) = case (evalPLambda e1, evalPLambda e2) of
%       (VLit x, VLit y) -> VLit (x + y)
%     evalPLambda (PMult e1 e2) = case (evalPLambda e1, evalPLambda e2) of
%       (VLit x, VLit y) -> VLit (x * y)
%     evalPLambda (PEq e1 e2) = case (evalPLambda e1, evalPLambda e2) of
%       (VLit x, VLit y) -> VBool (x == y)
%     evalPLambda (PLam f) = Val (evalPLambda . f)
%     evalPLambda (PApp e1 e2) = case evalPLambda e1 of
%       Val f -> f (evalPLambda e2)
% \end{minted}
% \caption{An evaluator for the PHOAS-encoded lambda calculus}
% \label{lst:evaluator}
% \end{listing}
